\documentclass{beamer}
\usepackage{ctex}
\usepackage{amsmath}
\usepackage{amssymb}
\usepackage{graphicx}
\usepackage{booktabs}
\usetheme{Madrid}
\definecolor{myteal}{cmyk}{0.5,0,0.15,0}
\definecolor{myyellow}{cmyk}{0,0.2,0.7,0}
\definecolor{myblue}{cmyk}{0.80,0.13,0.14,0.04}
\definecolor{mygreen}{cmyk}{0.4,0,0.4,0}

\setbeamercolor{structure}{fg=mygreen}                     % 所有结构性元素(项目符号、链接等)
\setbeamercolor{title}{fg=white, bg=mygreen}               % 主标题背景
\setbeamercolor{frametitle}{fg=white, bg=mygreen!90!black} % 帧标题(稍深一点增加层次)
\setbeamercolor{block title}{fg=white, bg=mygreen!80!black}% 定义/性质等 block 标题
\setbeamercolor{block body}{bg=mygreen!5}                  % block 内容背景(极淡青绿)
\setbeamercolor{itemize item}{fg=mygreen}                  % 项目符号颜色
\title{正则化与归纳偏置}
\institute{上海师范大学数理学院}
\date{}
\begin{document}
\frame{\titlepage}
\section{正则化基础}
\begin{frame}{正则化基础}
\begin{block}{正则化定义}
正则化是深度学习中控制模型复杂度、防止过拟合的核心技术。通过向误差函数添加惩罚项,形成正则化的误差函数:
\[
\tilde{E}(\mathbf{w}) = E(\mathbf{w}) + \frac{\lambda}{2} \mathbf{w}^T\mathbf{w}
\]
其中:
\begin{itemize}
\item $\mathbf{w}$ 是模型参数向量
\item $E(\mathbf{w})$ 是未正则化的误差函数
\item $\lambda$ 是正则化超参数,控制正则化强度
\end{itemize}
\end{block}

\begin{itemize}
\item 通过减少解的方差(以增加一些偏差为代价)提高预测准确性
\item 任意给定神经网络只能表示真实数据生成器的近似
\item 最佳泛化通常通过较大网络结合正则化获得
\item 多种正则化技术可以组合使用:早停、模型平均、Dropout、数据增强和参数共享等
\end{itemize}
\end{frame}

\section{归纳偏置}

\begin{frame}{模型复杂度与归纳偏置}
\begin{block}{偏差-方差权衡}
\begin{itemize}
\item 中等复杂度的模型通常获得最小泛化误差
\item 高偏差的简单模型无法捕捉数据生成过程中的变化
\item 低偏差的高度灵活模型易过拟合,泛化性能差
\item 随数据集增长,可使用更灵活模型(更少偏差)获得更好泛化
\end{itemize}
\end{block}

\begin{itemize}
\item 正则化系数$\lambda$的中间值通常对新输入给出最佳预测
\item 模型选择也受内存使用、执行速度等因素影响
\item 本讨论专注于实现良好预测性能这一核心目标
\item 适当水平的偏差对于从有限数据集中进行泛化非常重要
\end{itemize}
\end{frame}

\begin{frame}{逆问题与归纳偏置}
\begin{block}{机器学习作为逆问题}
\begin{itemize}
\item 机器学习任务本质上是逆问题:从有限样本推断整个分布
\item 无限多个分布可能生成观察到的训练数据
\item 从无限可能性中选择特定分布需要归纳偏置
\end{itemize}
\end{block}

\begin{itemize}
\item 归纳偏置源于对问题背景的先验知识
\item 常见偏置:输入微小变化应导致输出微小变化(平滑性)
\item 权重正则化鼓励权重小幅度,引入对平滑函数的偏置
\item 领域特定偏置:如图像中物体身份与位置无关(平移不变性)
\item 迁移学习和多任务学习也从正则化角度理解,任务间相似性是更强的归纳偏置
\end{itemize}
\end{frame}

\begin{frame}{没有免费午餐定理}
\begin{block}{Wolpert (1996)定理}
\begin{itemize}
\item 在所有可能问题上取平均,每个学习算法与其他算法一样好
\item 特定算法在某些问题上优于平均水平,必在其他问题上劣于平均水平
\end{itemize}
\end{block}

\begin{itemize}
\item 理论性限制:实际应用通常表现出平滑性(输入小变化→输出小变化)
\item 神经网络等模型包含对平滑性的偏置,具有广泛适用性
\item 归纳偏置的核心重要性:没有偏置,不可能"纯粹从数据中学习"
\item 偏置可以是隐含的(如有限参数数量)或显式编码的
\item 通用学习算法:适用于广泛实际应用的归纳偏置
\item 基于模型的机器学习:使所有假设显式化,为归纳偏置做适当选择
\end{itemize}
\end{frame}

\begin{frame}{对称性与不变性}
\begin{block}{输入变换下的不变性}
\begin{itemize}
\item 预测应在输入变量某些变换下保持不变
\item 例:图像物体分类中的平移不变性、尺度不变性
\item 变换保持特定属性不变,代表对称性
\item 对称变换集合形成群(group)数学结构
\end{itemize}
\end{block}

\begin{itemize}
\item 群公理:封闭性、结合律、单位元、逆元
\item 例:90度倍数旋转、二维平面连续平移
\item 从数据中学习不变性具有挑战性:变换导致数据显著变化
\item 需要更有效方法鼓励模型展现所需不变性
\end{itemize}
\end{frame}

\begin{frame}{对称性与不变性的实现}
\begin{block}{四种实现不变性的方法}
\begin{enumerate}
\item \textbf{预处理}:计算不变特征,后续系统必然尊重这些不变性
\item \textbf{正则化误差函数}:当输入受不变变换影响时,惩罚输出变化
\item \textbf{数据增强}:扩展训练集,包含变换版本,相同目标输出
\item \textbf{网络架构}:将不变性构建到网络结构中
\end{enumerate}
\end{block}

\begin{itemize}
\item 预处理挑战:设计不变特征但不丢失有用信息
\item 正则化例:切线传播技术,仅适用于小变换
\item 数据增强:易实现且高效,广泛用于图像分析(翻转、缩放、旋转、亮度变化等)
\item 网络架构:最强大有效,如卷积神经网络
\end{itemize}
\end{frame}

\begin{frame}{等变性}
\begin{block}{等变性定义}
\begin{itemize}
\item 不变性的推广:输出以特定方式变换,而非保持不变
\item 例:图像分割网络,物体平移→分割结果相应平移
\end{itemize}
\end{block}

\begin{itemize}
\item 平移等变性:$S(T(I)) = T(S(I))$
\begin{itemize}
\item $I$:图像,$S$:分割网络,$T$:平移操作
\item 平移后图像的分割 = 原始图像分割的平移
\end{itemize}
\item 一般等变性:$S(T(I)) = \tilde{T}(S(I))$
\begin{itemize}
\item 输出与输入变换不同
\item 例:分割图像分辨率较低时,$\tilde{T}$是低维空间中的平移
\end{itemize}
\item 不变性是等变性的特例:输出变换为单位变换
\begin{itemize}
\item 例:分类网络:$C(T(I)) = C(I)$
\end{itemize}
\end{itemize}
\end{frame}

\section{权重衰减}

\begin{frame}{权重衰减的基本原理}
\begin{block}{二次正则化}
\begin{itemize}
\item 最简单正则化器:参数平方和
\item 有效模型复杂度由正则化系数$\lambda$决定
\item 可解释为权重向量$\mathbf{w}$上零均值高斯先验分布的负对数
\end{itemize}
\end{block}

\begin{itemize}
\item 称为"权重衰减":在顺序学习中,鼓励权重向零衰减
\item 梯度计算简单:$\nabla\tilde{E}(\mathbf{w}) = \nabla E(\mathbf{w}) + \lambda\mathbf{w}$
\item 一般效果:缩小权重参数幅度,尤其对不敏感参数
\item 有效参数数量:正则化后保持活跃的参数数量
\begin{itemize}
\item $\lambda \to \infty$:所有参数为零,有效参数数量为零
\item $\lambda = 0$:有效参数数量等于实际参数总数
\end{itemize}
\end{itemize}
\end{frame}

\begin{frame}{一致性正则化器}
\begin{block}{变换不变性要求}
\begin{itemize}
\item 简单权重衰减破坏网络映射的理想变换属性
\item 例:两层MLP,输入/输出线性变换应保持映射不变
\item 一致性要求:等效解决方案仅在权重线性变换上不同
\end{itemize}
\end{block}

\begin{itemize}
\item 一致性正则化器:$\frac{\lambda_1}{2} \sum_{w \in W_1} w^2 + \frac{\lambda_2}{2} \sum_{w \in W_2} w^2$
\begin{itemize}
\item $W_1, W_2$:各层权重集合
\item 偏置不包括在求和中
\item 权重重新缩放时保持不变
\end{itemize}
\item 对应先验分布:$p(\mathbf{w}|\alpha_1, \alpha_2) \propto \exp(-\frac{\alpha_1}{2} \sum_{w \in W_1} w^2 - \frac{\alpha_2}{2} \sum_{w \in W_2} w^2)$
\item 一般形式:$\Omega(\mathbf{w}) = \frac{1}{2}\sum_k \alpha_k \|\mathbf{w}\|^2_k$,$\|\mathbf{w}\|^2_k = \sum_{j \in W_k} w_j^2$
\end{itemize}
\end{frame}

\begin{frame}{广义权重衰减}
\begin{block}{Lq正则化}
\begin{itemize}
\item 推广形式:$\Omega(\mathbf{w}) = \frac{\lambda}{2}\sum_{j=1}^M |w_j|^q$
\item $q=2$:二次正则化器(L2正则化)
\item $q=1$:lasso正则化器(Tibshirani, 1996)
\end{itemize}
\end{block}

\begin{itemize}
\item L1正则化特点:
\begin{itemize}
\item 使一些系数$w_j$被驱动为零,导致稀疏模型
\item 等价于在约束$\sum_{j=1}^M |w_j|^q \leq \eta$下最小化未正则化误差
\end{itemize}
\item 稀疏性起源:L1约束区域在坐标轴上有尖角,更可能与误差函数最小值相交
\item 正则化允许在有限数据集上训练复杂模型
\item 问题从找到适当参数数量转变为确定合适正则化系数$\lambda$
\end{itemize}
\end{frame}

\section{学习曲线}

\begin{frame}{早停}
\begin{block}{训练过程中的泛化}
\begin{itemize}
\item 训练误差通常随迭代单调减少
\item 验证误差先减少后增加(过拟合)
\item 早停:在验证误差最小时停止训练
\end{itemize}
\end{block}

\begin{itemize}
\item 有效参数数量解释:
\begin{itemize}
\item 有效参数数量在训练过程中增长
\item 限制训练迭代次数限制有效网络复杂度
\end{itemize}
\item 二次误差函数下,早停与权重衰减行为相似
\begin{itemize}
\item 量$\tau\eta$($\tau$:迭代索引,$\eta$:学习率)类似$\lambda^{-1}$
\end{itemize}
\item 早停是控制模型复杂度的实用方法
\end{itemize}
\end{frame}

\begin{frame}{双下降现象}
\begin{block}{模型复杂度与泛化}
\begin{itemize}
\item 传统观点:参数过多导致过拟合,泛化性能下降
\item 深度学习经验:更大模型通常表现更好
\item 双下降现象(Belkin等,2019):
\begin{itemize}
\item 测试误差先下降,后增加,再下降
\end{itemize}
\end{itemize}
\end{block}

\begin{itemize}
\item 两种模式:
\begin{itemize}
\item 中小型复杂度:经典偏差-方差权衡
\item 非常大模型:测试误差在训练误差零后继续减少
\end{itemize}
\item 有效模型复杂度:模型达到零训练误差的最大训练数据集大小
\item 早停和正则化也显示双下降行为
\item 讽刺结果:增加训练数据可能降低性能(临界模式)
\end{itemize}
\end{frame}

\section{参数共享}

\begin{frame}{硬参数共享}
\begin{block}{参数共享原理}
\begin{itemize}
\item 通过对权重施加硬约束,将权重分组
\item 每组中所有权重共享相同值,共享值从数据中学习
\item 自由度数量小于网络连接数量
\end{itemize}
\end{block}

\begin{itemize}
\item 作为将归纳偏置编码到网络中的方式
\item 表达已知不变性(如平移不变性)
\item 梯度评估可通过修改反向传播完成
\item 卷积神经网络中大量使用参数共享
\item 仅适用于可提前指定约束形式的问题
\end{itemize}
\end{frame}

\begin{frame}{软权重共享}
\begin{block}{高斯混合正则化}
\begin{itemize}
\item Nowlan和Hinton (1992)提出:软权重共享
\item 不强制权重相等,鼓励组内权重值相似
\item 权重分组、平均值和分布作为学习过程一部分确定
\end{itemize}
\end{block}

\begin{itemize}
\item 先验分布:$p(\mathbf{w}) = \prod_i \{\sum_{j=1}^K \pi_j \mathcal{N}(w_i|\mu_j, \sigma_j^2)\}$
\item 正则化函数:$\Omega(\mathbf{w}) = -\sum_i \ln(\sum_{j=1}^K \pi_j \mathcal{N}(w_i|\mu_j, \sigma_j^2))$
\item 后验概率:$\gamma_j(w_i) = \frac{\pi_j \mathcal{N}(w_i|\mu_j, \sigma_j^2)}{\sum_k \pi_k \mathcal{N}(w_i|\mu_k, \sigma_k^2)}$
\item 权重导数:$\frac{\partial \tilde{E}}{\partial w_i} = \frac{\partial E}{\partial w_i} + \lambda \sum_j \gamma_j(w_i) \frac{(w_i - \mu_j)}{\sigma_j^2}$
\item 其他参数($\mu_j$, $\sigma_j^2$, $\pi_j$)也有相应更新规则
\end{itemize}
\end{frame}

\section{残差连接}

\begin{frame}{残差连接的动机}
\begin{block}{深度网络训练挑战}
\begin{itemize}
\item 破碎梯度现象(Balduzzi等,2017):
\begin{itemize}
\item 深度增加→线性区域数量指数增长
\item 误差函数梯度不连续性增殖
\item 早期层权重微小变化→梯度显著变化
\end{itemize}
\item 梯度优化算法假设梯度平滑变化
\end{itemize}
\end{block}

\begin{itemize}
\item 残差连接(He等,2015a):特定形式的跳跃层连接
\item 使训练非常深网络(数百层)成为可能
\item 与批归一化结合,显著减少梯度消失/爆炸问题
\end{itemize}
\end{frame}

\begin{frame}{残差连接实现}
\begin{block}{残差块}
\begin{itemize}
\item 标准层:$\mathbf{z}_l = F_l(\mathbf{z}_{l-1})$
\item 残差连接:$\mathbf{z}_l = F_l(\mathbf{z}_{l-1}) + \mathbf{z}_{l-1}$
\item $F_l(\mathbf{z}_{l-1}) = \mathbf{z}_l - \mathbf{z}_{l-1}$:学习恒等映射与期望输出间的残差
\item 残差块可轻松生成恒等变换
\end{itemize}
\end{block}

\begin{itemize}
\item 残差网络效果:
\begin{itemize}
\item 梯度对输入值敏感度小得多
\item 创建更平滑的误差函数表面
\end{itemize}
\item 扩展形式:多个并行工作的子网络(不同深度)
\item 维度变化:$\mathbf{z}_l = F_l(\mathbf{z}_{l-1}) + \mathbf{W}\mathbf{z}_{l-1}$
\item 常见配置:跳跃连接起点$\rightarrow \mathbf{ReLU} \rightarrow \mathbf{W} \rightarrow$跳跃连接终点
\end{itemize}
\end{frame}


\section{模型平均}

\begin{frame}{模型平均基础}
\begin{block}{committee方法}
\begin{itemize}
\item 多个模型预测平均:$p(y|\mathbf{x}) = \frac{1}{L} \sum_{l=1}^L p_l(y|\mathbf{x})$
\item 通过偏差-方差权衡改进预测
\item 需要在committee成员间引入可变性
\end{itemize}
\end{block}

\begin{itemize}
\item 自助聚合(Bagging):
\begin{itemize}
\item 创建多个自助数据集:从原始数据集有放回抽样
\item 每个数据集训练一个模型
\item 不同架构的多个模型也可形成集成
\end{itemize}
\item 误差分析:
\begin{itemize}
\item committee误差:$E_{COM} = \mathbb{E}_{\mathbf{x}}[\{\frac{1}{M}\sum_{m=1}^M \epsilon_m(\mathbf{x})\}^2]$
\item 不相关误差时:$E_{COM} = \frac{1}{M}E_{AV}$
\item 现实中误差相关,但$E_{COM} \leq E_{AV}$始终成立
\end{itemize}
\end{itemize}
\end{frame}

\begin{frame}{Boosting:训练与预测}
\begin{block}{训练过程}
\begin{itemize}
\item \textbf{顺序训练}:基分类器一个接一个训练
\item \textbf{样本加权}:每轮增加前一轮错误分类样本的权重
\item \textbf{分类器加权}:性能好的分类器获得更高投票权重
\end{itemize}
\end{block}

\begin{block}{预测过程}
\begin{itemize}
\item \textbf{加权投票}:$H(\mathbf{x}) = \text{sign}\left(\sum_{t=1}^T \alpha_t h_t(\mathbf{x})\right)$
\item 每个基分类器的投票权重$\alpha_t$与其准确率成正比
\end{itemize}
\end{block}

\textbf{关键特性}:
\begin{itemize}
\item 弱分类器只需略优于随机猜测
\item 与Bagging不同:训练是顺序的,非并行
\item 适合处理难例样本,降低偏差
\end{itemize}
\end{frame}

\begin{frame}{Boosting vs Bagging 与实际应用}
\begin{block}{Boosting与Bagging对比}
\begin{table}
\centering
\begin{tabular}{lcc}
\toprule
\textbf{特性} & \textbf{Boosting} & \textbf{Bagging} \\
\midrule
训练方式 & 顺序 & 并行 \\
样本权重 & 动态调整 & 均匀/固定 \\
关注焦点 & 难例样本 & 减少方差 \\
过拟合风险 & 较高 & 较低 \\
基础模型 & 弱模型 & 任意模型 \\
\bottomrule
\end{tabular}
\end{table}
\end{block}

\textbf{计算代价考量}:
\begin{itemize}
\item 主要缺点:多个模型需分别训练,预测需评估所有模型
\item 训练和推理计算成本显著增加
\item 在资源受限场景,需权衡性能提升与计算开销
\item Dropout提供了近似模型平均的低成本替代方案
\end{itemize}
\end{frame}
\begin{frame}{Dropout:原理与训练机制}
\begin{block}{核心思想}
\begin{itemize}
\item \textbf{训练阶段}:随机删除网络节点(包括连接)
\item \textbf{每次迭代}:为每个样本生成新的随机掩码
\item \textbf{实现}:二元掩码向量$\mathbf{R} \in \{0,1\}^n$,节点保留概率$\rho$
\begin{itemize}
\item 隐藏层:$\rho = 0.5$
\item 输入层:$\rho = 0.8$
\end{itemize}
\item \textbf{不应用于}输出层
\end{itemize}
\end{block}

\begin{block}{训练过程}
\begin{itemize}
\item 每个mini-batch生成新掩码
\item 前向/反向传播仅在保留节点进行
\item 梯度更新基于掩码网络
\item 本质:训练指数级($2^M$)子网络的共享参数集合
\end{itemize}
\end{block}

\end{frame}

\begin{frame}{Dropout:预测与理论解释}
\begin{block}{预测方法}
\begin{itemize}
\item \textbf{蒙特卡洛方法}:
\begin{itemize}
\item 采样10-20个掩码
\item 平均预测结果
\end{itemize}
\item \textbf{权重缩放}(标准做法):
\begin{itemize}
\item 测试时使用完整网络
\item 每层权重 $\times\,\rho$
\item 保持期望激活值一致
\end{itemize}
\end{itemize}
\end{block}

\begin{block}{理论解释}
\begin{itemize}
\item \textbf{模型平均}:近似$2^M$个子网络的贝叶斯平均
\item \textbf{正则化机制}:
\begin{itemize}
\item 防止节点过度专业化
\item 减少共同适应(co-adaptation)
\item 增强特征鲁棒性
\end{itemize}
\end{itemize}
\end{block}

\end{frame}

\end{document}
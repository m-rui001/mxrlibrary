\documentclass{beamer}
\usepackage{ctex}
\usepackage{amsmath}
\usepackage{amssymb}
\usepackage{graphicx}
\usepackage{booktabs}
\usepackage[UTF8]{ctex}
\usetheme{Madrid}
\definecolor{myteal}{cmyk}{0.5,0,0.15,0}
\definecolor{myyellow}{cmyk}{0,0.2,0.7,0}
\definecolor{myblue}{cmyk}{0.80,0.13,0.14,0.04}
\definecolor{mygreen}{cmyk}{0.4,0,0.4,0}

\setbeamercolor{structure}{fg=mygreen}                     % 所有结构性元素(项目符号、链接等)
\setbeamercolor{title}{fg=white, bg=mygreen}               % 主标题背景
\setbeamercolor{frametitle}{fg=white, bg=mygreen!90!black} % 帧标题(稍深一点增加层次)
\setbeamercolor{block title}{fg=white, bg=mygreen!80!black}% 定义/性质等 block 标题
\setbeamercolor{block body}{bg=mygreen!5}                  % block 内容背景(极淡青绿)
\setbeamercolor{itemize item}{fg=mygreen}                  % 项目符号颜色
\title{单层神经网络:回归分析}
\institute{上海师范大学数理学院}
\date{}
\begin{document}
\frame{\titlepage}

\section{线性回归模型}

\begin{frame}{线性回归基础}
\begin{block}{回归问题定义}
回归任务的目标是预测一个或多个连续目标变量 $t$ 的值,给定一个 $D$ 维输入向量 $\mathbf{x}$。我们通常拥有包含 $N$ 个观测样本 $\{\mathbf{x}_n\}$ 及对应目标值 $\{t_n\}$ 的训练数据集,目标是预测新输入 $\mathbf{x}$ 对应的 $t$ 值。
\end{block}

\begin{itemize}
\item 核心思想:构建函数 $y(\mathbf{x}, \mathbf{w})$,其输出作为目标变量 $t$ 的预测
\item 参数 $\mathbf{w}$ 通过训练数据学习获得
\item 最简单模型:输入变量的线性组合
\[
y(\mathbf{x}, \mathbf{w}) = w_0 + w_1x_1 + \cdots + w_Dx_D
\]
\item 关键特性:模型关于参数 $\{w_0,\ldots,w_D\}$ 是线性的
\item 局限性:也是输入变量 $\{x_i\}$ 的线性函数,这严重限制了模型表达能力
\end{itemize}
\end{frame}

\begin{frame}{基础函数扩展}
\begin{block}{非线性变换}
通过使用固定非线性函数的线性组合,可以扩展线性模型的能力:
\[
y(\mathbf{x}, \mathbf{w}) = w_0 + \sum_{j=1}^{M-1} w_j\phi_j(\mathbf{x})
\]
其中 $\phi_j(\mathbf{x})$ 称为基础函数,$M$ 是总参数数量。
\end{block}

\begin{itemize}
\item \textbf{偏置参数} $w_0$:允许数据中存在固定偏移
\item 为简化表示,常定义虚拟基础函数 $\phi_0(\mathbf{x}) = 1$,使模型变为:
\[
y(\mathbf{x}, \mathbf{w}) = \sum_{j=0}^{M-1} w_j\phi_j(\mathbf{x}) = \mathbf{w}^T\phi(\mathbf{x})
\]
其中 $\mathbf{w} = (w_0,\ldots,w_{M-1})^T$ 且 $\phi = (\phi_0,\ldots,\phi_{M-1})^T$
\item \textbf{神经网络表示}:该模型可表示为单层神经网络
\item 尽管使用非线性基础函数,但模型仍被称为线性模型
\end{itemize}
\end{frame}

\begin{frame}{常用基础函数类型}
\begin{block}{多样化基础函数}
选择合适的基础函数对模型性能至关重要。常见类型包括:
\end{block}

\begin{itemize}
\item \textbf{多项式基础函数}:$\phi_j(x) = x^j$
\begin{itemize}
\item 简单直观,但在高维空间效果不佳
\item 高阶项可能导致数值不稳定
\end{itemize}

\item \textbf{高斯基础函数}:
\[
\phi_j(\mathbf{x}) = \exp\left\{-\frac{(\mathbf{x}-\boldsymbol{\mu}_j)^2}{2s^2}\right\}
\]
\item \textbf{Sigmoid基础函数}:
\[
\phi_j(\mathbf{x}) = \sigma\left(\frac{\mathbf{x}-\boldsymbol{\mu}_j}{s}\right)
\]
其中 $\sigma(a) = \frac{1}{1+\exp(-a)}$ 为逻辑sigmoid函数
\begin{itemize}
\item tanh函数与sigmoid等价:$\tanh(a) = 2\sigma(2a)-1$
\end{itemize}
\end{itemize}
\end{frame}

\begin{frame}{基础函数特性与选择}
\begin{block}{基础函数的频域特性}
不同基础函数在空间和频率域表现出不同特性:
\end{block}

\begin{itemize}
\item \textbf{傅里叶基础}:正弦和余弦函数
\begin{itemize}
\item 每个基础函数代表特定频率
\item 具有无限空间范围
\item 适合周期性数据
\end{itemize}

\item \textbf{小波基础}:
\begin{itemize}
\item 同时在空间和频率域局部化
\item 通常定义为相互正交
\item 特别适用于输入值位于规则网格上的情况(如时间序列、图像像素)
\end{itemize}

\item \textbf{现代深度学习方法}:
\begin{itemize}
\item 传统方法需要手工设计基础函数,这在复杂应用中非常困难
\item 深度学习避免了这一问题,通过从数据本身学习所需的非线性变换
\item 网络层自动学习适合任务的基础表示
\end{itemize}
\end{itemize}
\end{frame}

\section{概率视角}

\begin{frame}{似然函数与高斯噪声}
\begin{block}{概率模型}
假设目标变量 $t$ 由确定性函数 $y(\mathbf{x},\mathbf{w})$ 加上高斯噪声构成:
\[
t = y(\mathbf{x},\mathbf{w}) + \varepsilon
\]
其中 $\varepsilon$ 是零均值、方差为 $\sigma^2$ 的高斯随机变量。
\end{block}

\begin{itemize}

\item 给定独立同分布数据集 $\{\mathbf{x}_n,t_n\}_{n=1}^N$,似然函数为:
\[
p(\mathbf{t}|\mathbf{X},\mathbf{w},\sigma^2) = \prod_{n=1}^N \mathcal{N}(t_n|\mathbf{w}^T\phi(\mathbf{x}_n),\sigma^2)
\]
其中 $\mathbf{t} = (t_1,\ldots,t_N)^T$,$\mathbf{X} = \{\mathbf{x}_1,\ldots,\mathbf{x}_N\}$

\item 对数似然函数:
\[
\ln p(\mathbf{t}|\mathbf{X},\mathbf{w},\sigma^2) = -\frac{N}{2}\ln\sigma^2 - \frac{N}{2}\ln(2\pi) - \frac{1}{\sigma^2}E_D(\mathbf{w})
\]
其中 $E_D(\mathbf{w}) = \frac{1}{2}\sum_{n=1}^N\{t_n - \mathbf{w}^T\phi(\mathbf{x}_n)\}^2$ 为平方和误差函数
\end{itemize}
\end{frame}

\begin{frame}{最大似然估计}
\begin{block}{参数优化}
最大化对数似然函数等价于最小化平方和误差函数,因为前两项与 $\mathbf{w}$ 无关。
\end{block}

\begin{itemize}
\item \textbf{权重参数} $\mathbf{w}$ 的梯度:
\[
\nabla_{\mathbf{w}}\ln p(\mathbf{t}|\mathbf{X},\mathbf{w},\sigma^2) = \frac{1}{\sigma^2}\sum_{n=1}^N\{t_n - \mathbf{w}^T\phi(\mathbf{x}_n)\}\phi(\mathbf{x}_n)^T
\]

\item 令梯度为零,解得:
\[
\mathbf{w}_{\text{ML}} = (\Phi^T\Phi)^{-1}\Phi^T\mathbf{t}
\]
这称为最小二乘问题的\textbf{正规方程}

\item \textbf{设计矩阵} $\Phi$:$N\times M$ 矩阵,元素 $\Phi_{nj} = \phi_j(\mathbf{x}_n)$

\item 矩阵 $\Phi^\dagger \equiv (\Phi^T\Phi)^{-1}\Phi^T$ 称为\textbf{Moore-Penrose伪逆},是矩阵逆概念对非方阵的推广
\end{itemize}
\end{frame}

\begin{frame}{偏差参数与噪声估计}
\begin{itemize}
\item 令 $\partial E_D/\partial w_0 = 0$,解得:
\[
w_0 = \bar{t} - \sum_{j=1}^{M-1}w_j\bar{\phi}_j
\]
其中 $\bar{t} = \frac{1}{N}\sum_{n=1}^N t_n$,$\bar{\phi}_j = \frac{1}{N}\sum_{n=1}^N\phi_j(\mathbf{x}_n)$

\item 物理解释:偏置参数 $w_0$ 补偿了目标值平均值与加权基础函数平均值之间的差异

\item \textbf{噪声方差估计}:对 $\sigma^2$ 最大化对数似然
\[
\sigma^2_{\text{ML}} = \frac{1}{N}\sum_{n=1}^N\{t_n - \mathbf{w}_{\text{ML}}^T\phi(\mathbf{x}_n)\}^2
\]

\item 解释:最大似然估计的方差参数等于目标值围绕回归函数的残差方差
\end{itemize}
\end{frame}

\begin{frame}{最小二乘的几何解释}
\begin{block}{向量空间视角}
考虑 $N$ 维空间,坐标轴对应 $t_1,\ldots,t_N$,则 $\mathbf{t} = (t_1,\ldots,t_N)^T$ 是该空间中的向量。
\end{block}

\begin{itemize}
\item 每个基础函数 $\phi_j(\mathbf{x})$ 在 $N$ 个数据点上的取值构成向量 $\boldsymbol{\phi}_j$
\begin{itemize}
\item $\boldsymbol{\phi}_j$ 对应设计矩阵 $\Phi$ 的第 $j$ 列
\item $\phi(\mathbf{x}_n)$ 对应 $\Phi$ 的第 $n$ 行转置
\end{itemize}

\item 若基础函数数量 $M$ 小于数据点数量 $N$,则 $M$ 个向量 $\boldsymbol{\phi}_j$ 张成 $M$ 维线性子空间 $S$
\item 定义 $N$ 维向量 $\mathbf{y}$,其第 $n$ 个元素为 $y(\mathbf{x}_n,\mathbf{w})$
\begin{itemize}
\item $\mathbf{y}$ 是 $\boldsymbol{\phi}_j$ 的任意线性组合,可位于子空间 $S$ 中任意位置
\end{itemize}

\item 平方和误差 $E_D(\mathbf{w})$ 等于 $\mathbf{y}$ 与 $\mathbf{t}$ 之间欧氏距离平方的一半
\item 最小二乘解对应于 $\mathbf{t}$ 在子空间 $S$ 上的\textbf{正交投影}
\end{itemize}
\end{frame}

\section{优化与正则化}

\begin{frame}{序列学习算法}
\begin{block}{批量方法与在线方法}
最大似然解 $\mathbf{w}_{\text{ML}} = (\Phi^T\Phi)^{-1}\Phi^T\mathbf{t}$ 涉及一次性处理整个训练集,称为\textbf{批量方法}。
\end{block}

\begin{itemize}
\item \textbf{计算挑战}:
\begin{itemize}
\item 大型数据集上计算成本高昂
\item 实时应用中数据可能连续到达
\end{itemize}

\item \textbf{随机梯度下降} (SGD):
\begin{itemize}
\item 适用于可分解为数据点和的误差函数 $E = \sum_n E_n$
\item 每次呈现一个数据点后更新参数
\item 更新规则:$\mathbf{w}^{(\tau+1)} = \mathbf{w}^{(\tau)} - \eta\nabla E_n$
\end{itemize}

\item \textbf{LMS算法}(最小均方算法):
\begin{itemize}
\item 针对平方和误差的SGD
\item $\mathbf{w}^{(\tau+1)} = \mathbf{w}^{(\tau)} + \eta(t_n - \mathbf{w}^{(\tau)T}\phi_n)\phi_n$
\item 其中 $\phi_n = \phi(\mathbf{x}_n)$,$\eta$ 为学习率
\item $\mathbf{w}$ 从初始向量 $\mathbf{w}^{(0)}$ 开始迭代
\end{itemize}
\end{itemize}
\end{frame}

\begin{frame}{正则化最小二乘}
\begin{itemize}
\item \textbf{权重衰减}:最简单正则化形式
\[
E_W(\mathbf{w}) = \frac{1}{2}\sum_j w_j^2 = \frac{1}{2}\mathbf{w}^T\mathbf{w}
\]

\item 总误差函数:
\[
\frac{1}{2}\sum_{n=1}^N\{t_n - \mathbf{w}^T\phi(\mathbf{x}_n)\}^2 + \frac{\lambda}{2}\mathbf{w}^T\mathbf{w}
\]

\item \textbf{统计解释}:参数收缩方法,将参数值向零收缩
\item 优势:误差函数保持关于 $\mathbf{w}$ 的二次形式,可得到闭式解:
\[
\mathbf{w} = (\lambda I + \Phi^T\Phi)^{-1}\Phi^T\mathbf{t}
\]

\item \textbf{数值稳定性}:正则化项确保矩阵 $(\lambda I + \Phi^T\Phi)$ 非奇异,即使在基础向量接近共线时
\item 当 $\lambda \to 0$ 时,解趋近于未正则化的最小二乘解
\end{itemize}
\end{frame}

\begin{frame}{多输出回归}
\begin{block}{多变量预测}
有时需要同时预测 $K>1$ 个目标变量,记为向量 $\mathbf{t} = (t_1,\ldots,t_K)^T$。
\end{block}

\begin{itemize}
\item \textbf{多输出回归}:使用相同基础函数集建模所有目标分量
\[
\mathbf{y}(\mathbf{x},\mathbf{W}) = \mathbf{W}^T\phi(\mathbf{x})
\]
\begin{itemize}
\item $\mathbf{y}$ 是 $K$ 维列向量
\item $\mathbf{W}$ 是 $M \times K$ 参数矩阵
\item $\phi(\mathbf{x})$ 是 $M$ 维列向量
\end{itemize}

\item \textbf{神经网络表示}:单层参数网络,多个输出节点

\item 假设条件分布为各向同性高斯:
\[
p(\mathbf{t}|\mathbf{x},\mathbf{W},\sigma^2) = \mathcal{N}(\mathbf{t}|\mathbf{W}^T\phi(\mathbf{x}),\sigma^2I)
\]

\item 最大似然解:$\mathbf{W}_{\text{ML}} = (\Phi^T\Phi)^{-1}\Phi^T\mathbf{T}$
\begin{itemize}
\item $\mathbf{T}$ 是 $N \times K$ 矩阵,每行对应一个目标向量的转置
\item 该解在不同目标变量间解耦,只需计算一次伪逆矩阵 $\Phi^\dagger$
\end{itemize}
\end{itemize}
\end{frame}

\section{决策理论}

\begin{frame}{回归的决策理论}
\begin{block}{两阶段预测框架}
回归任务可分解为两个阶段:
\end{block}

\begin{enumerate}
\item \textbf{推断阶段}:使用训练数据确定预测分布 $p(t|\mathbf{x})$
\item \textbf{决策阶段}:使用预测分布确定最优预测值 $f(\mathbf{x})$
\end{enumerate}

\begin{itemize}
\item 概率模型:假设条件分布为高斯分布
\[
p(t|\mathbf{x},\mathbf{w},\sigma^2) = \mathcal{N}(t|y(\mathbf{x},\mathbf{w}),\sigma^2)
\]

\item 预测分布:
\[
p(t|\mathbf{x},\mathbf{w}_{\text{ML}},\sigma^2_{\text{ML}}) = \mathcal{N}(t|y(\mathbf{x},\mathbf{w}_{\text{ML}}),\sigma^2_{\text{ML}})
\]

\item \textbf{损失函数}:衡量预测 $f(\mathbf{x})$ 与真实值 $t$ 之间的误差
\begin{itemize}
\item 选择特定 $f(\mathbf{x})$ 会带来损失 $L(t,f(\mathbf{x}))$
\item 由于不知道 $t$ 的真实值,我们最小化期望损失
\end{itemize}

\item 期望损失:
\[
\mathbb{E}[L] = \iint L(t,f(\mathbf{x}))p(\mathbf{x},t) d\mathbf{x}dt
\]
\end{itemize}
\end{frame}

\begin{frame}{平方损失与回归函数}
\begin{block}{最优决策}
平方损失函数 $L(t,f(\mathbf{x})) = \{f(\mathbf{x})-t\}^2$ 是回归问题的常用选择。
\end{block}

\begin{itemize}
\item 期望损失:
\[
\mathbb{E}[L] = \iint \{f(\mathbf{x})-t\}^2p(\mathbf{x},t) d\mathbf{x}dt
\]

\item \textbf{变分法求解}:对任意灵活函数 $f(\mathbf{x})$ 最小化 $\mathbb{E}[L]$
\[
\frac{\delta\mathbb{E}[L]}{\delta f(\mathbf{x})} = 2\int\{f(\mathbf{x})-t\}p(\mathbf{x},t)dt = 0
\]

\item 最优解(回归函数):
\[
f^\star(\mathbf{x}) = \frac{1}{p(\mathbf{x})}\int tp(\mathbf{x},t)dt = \int tp(t|\mathbf{x})dt = \mathbb{E}_t[t|\mathbf{x}]
\]

\item 物理解释:回归函数是最优预测,即条件分布 $p(t|\mathbf{x})$ 的均值


\item 对于高斯条件分布,条件均值简化为:
\[
\mathbb{E}[t|\mathbf{x}] = \int tp(t|\mathbf{x})dt = y(\mathbf{x},\mathbf{w})
\]
\end{itemize}
\end{frame}

\begin{frame}{损失函数分析}
\begin{block}{期望损失分解}
利用条件期望性质,平方损失期望可分解为:
\[
\{f(\mathbf{x})-t\}^2 = \{f(\mathbf{x})-\mathbb{E}[t|\mathbf{x}]+\mathbb{E}[t|\mathbf{x}]-t\}^2
\]
\end{block}

\begin{itemize}
\item 展开后代入期望损失:
\[
\mathbb{E}[L] = \int\{f(\mathbf{x})-\mathbb{E}[t|\mathbf{x}]\}^2p(\mathbf{x})d\mathbf{x} + \int\text{var}[t|\mathbf{x}]p(\mathbf{x})d\mathbf{x}
\]

\item 解释:
\begin{itemize}
\item 第一项:预测函数 $f(\mathbf{x})$ 与条件期望间的差异
\item 第二项:条件分布的方差,平均后表示数据的内在变异性(噪声)
\end{itemize}

\item \textbf{关键结论}:
\begin{itemize}
\item 当 $f(\mathbf{x}) = \mathbb{E}[t|\mathbf{x}]$ 时,第一项为零,达到最小值
\item 第二项与 $f(\mathbf{x})$ 无关,代表不可减少的最小损失
\item 最优最小二乘预测器由条件均值给出
\end{itemize}

\item \textbf{推广}:更一般的Minkowski损失 $\mathbb{E}[L_q] = \iint\|f(\mathbf{x})-t\|^qp(\mathbf{x},t)d\mathbf{x}dt$
\end{itemize}
\end{frame}

\section{偏差-方差权衡}

\begin{frame}{模型复杂度与泛化}
\begin{block}{挑战与权衡}
线性回归模型中,基础函数的形式和数量通常给定。最大似然可能导致有限数据集上的严重过拟合,但限制基础函数数量会降低模型捕捉数据中重要趋势的灵活性。
\end{block}

\begin{itemize}
\item \textbf{核心问题}:如何确定合适的正则化系数 $\lambda$?
\begin{itemize}
\item 同时最小化正则化误差函数和 $\lambda$ 会导致未正则化解($\lambda=0$)
\item 需要独立于参数优化的方法
\end{itemize}

\item \textbf{频率学派视角}:偏差-方差权衡
\begin{itemize}
\item 考虑条件期望 $h(\mathbf{x}) = \mathbb{E}[t|\mathbf{x}] = \int tp(t|\mathbf{x})dt$
\item 期望平方损失:
\[
\mathbb{E}[L] = \int\{f(\mathbf{x})-h(\mathbf{x})\}^2p(\mathbf{x})d\mathbf{x} + \iint\{h(\mathbf{x})-t\}^2p(\mathbf{x},t)d\mathbf{x}dt
\]
\item 第二项是数据固有噪声,与 $f(\mathbf{x})$ 无关
\end{itemize}

\item \textbf{贝叶斯视角}:过拟合是最大似然的不幸特性
\begin{itemize}
\item 通过参数边缘化,贝叶斯方法可避免过拟合问题
\end{itemize}
\end{itemize}
\end{frame}

\begin{frame}{偏差-方差分解}
\begin{block}{期望误差分析}
在有限数据集 $\mathcal{D}$(含 $N$ 个点)上,无法精确获知回归函数 $h(\mathbf{x})$。
\end{block}

\begin{itemize}
\item {}假设有大量大小为 $N$ 的数据集,每个都独立抽样自 $p(t,\mathbf{x})$
\begin{itemize}
\item 每个数据集 $\mathcal{D}$ 产生预测函数 $f(\mathbf{x};\mathcal{D})$
\item 不同数据集产生不同函数和损失值
\item 通过数据集集合上的平均评估学习算法性能
\end{itemize}

\item {}对特定输入 $\mathbf{x}$ 和数据集 $\mathcal{D}$,考虑
\begin{align*}
\mathbb{E}_\mathcal{D}\big[(f(\mathbf{x};\mathcal{D}) - h(\mathbf{x}))^2\big] 
&= \big(\mathbb{E}_\mathcal{D}[f(\mathbf{x};\mathcal{D})] - h(\mathbf{x})\big)^2 \\
&\quad + \mathbb{E}_\mathcal{D}\big[(f(\mathbf{x};\mathcal{D}) - \mathbb{E}_\mathcal{D}[f(\mathbf{x};\mathcal{D})])^2\big]
\end{align*}
\begin{itemize}
\item \textbf{平方偏差}:$\{\mathbb{E}_\mathcal{D}[f(\mathbf{x};\mathcal{D})]-h(\mathbf{x})\}^2$
\begin{itemize}
\item 衡量所有数据集平均预测与理想回归函数的差异
\end{itemize}
\item \textbf{方差}:$\mathbb{E}_\mathcal{D}[\{f(\mathbf{x};\mathcal{D})-\mathbb{E}_\mathcal{D}[f(\mathbf{x};\mathcal{D})]\}^2]$
\begin{itemize}
\item 衡量单个数据集解围绕平均解的波动程度
\item 反映函数 $f(\mathbf{x};\mathcal{D})$ 对特定数据集选择的敏感度
\end{itemize}
\end{itemize}
\end{itemize}
\end{frame}

\begin{frame}{期望损失分解}
\begin{block}{完整分解}
将偏差-方差分解整合到期望损失中,得到:
\[
\text{期望损失} = \text{(偏差)}^2 + \text{方差} + \text{噪声}
\]
\end{block}

\begin{align*}
\text{(偏差)}^2 &= \int\{\mathbb{E}_\mathcal{D}[f(\mathbf{x};\mathcal{D})]-h(\mathbf{x})\}^2p(\mathbf{x})d\mathbf{x} \\
\text{方差} &= \int\mathbb{E}_\mathcal{D}[\{f(\mathbf{x};\mathcal{D})-\mathbb{E}_\mathcal{D}[f(\mathbf{x};\mathcal{D})]\}^2]p(\mathbf{x})d\mathbf{x} \\
\text{噪声} &= \iint\{h(\mathbf{x})-t\}^2p(\mathbf{x},t)d\mathbf{x}dt
\end{align*}

\begin{itemize}
\item 高度灵活模型:低偏差,高方差
\item 相对刚性模型:高偏差,低方差
\item 最佳预测能力:在偏差和方差间取得平衡
\end{itemize}
\end{frame}
\end{document}

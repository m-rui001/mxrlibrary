\documentclass{article}
\usepackage{amsmath, amssymb, amsthm}
\usepackage[UTF8]{ctex}
\begin{document}
\newtheorem{theorem}{定理}
\begin{theorem}
设$f\in C[0,+\infty)\cap D^2(0,+\infty)$且存在常数$C>0$使得$f''(x)\leq\frac{C}{x^2}$对所有$x>0$成立。则有
$$
\lim_{x\to 0^+} xf'(x)=0
$$
\end{theorem}

\begin{proof}
对任意固定的$\eta \in (0,1)$和$x>0$,我们使用Taylor中值定理。

首先,在区间$(x-\eta x,x)$上,存在$\theta \in (x-\eta x,x)$使得:
\[
f(x-\eta x)=f(x)-\eta xf'(x)+(\eta x)^2\frac{f''(\theta)}{2}
\]

从中解出$f'(x)$:
\[
f'(x)=\frac{f(x)-f(x-\eta x)}{\eta x}+(\eta x)\frac{f''(\theta)}{2}
\]

由于$f''(\theta) \leq \frac{C}{\theta^2}$且$\theta > x-\eta x = x(1-\eta)$,有$\theta^2 > x^2(1-\eta)^2$,因此:
\[
f'(x) \leq \frac{f(x)-f(x-\eta x)}{\eta x}+(\eta x)\frac{C}{2\theta^2}
\]

两边乘以$x$:
\begin{align*}
xf'(x) &\leq \frac{f(x)-f(x-\eta x)}{\eta} + (\eta x^2)\frac{C}{2\theta^2} \\
&\leq \frac{f(x)-f(x-\eta x)}{\eta} + \frac{C\eta}{2} \cdot \frac{1}{(1-\eta)^2} \\
&= \frac{f(x)-f(x-\eta x)}{\eta} + \frac{C\eta}{2(1-\eta)^2}
\end{align*}

其次,在区间$(x,x+\eta x)$上,存在$\vartheta \in (x,x+\eta x)$使得:
\[
f(x+\eta x)=f(x)+\eta xf'(x)+(\eta x)^2\frac{f''(\vartheta)}{2}
\]

从中解出$f'(x)$:
\[
f'(x)=\frac{f(x+\eta x)-f(x)}{\eta x}-(\eta x)\frac{f''(\vartheta)}{2}
\]

由于$f''(\vartheta) \leq \frac{C}{\vartheta^2}$且$\vartheta > x$,有$\vartheta^2 > x^2$,因此:
\[
f'(x) \geq \frac{f(x+\eta x)-f(x)}{\eta x}-(\eta x)\frac{C}{2x^2}
\]

两边乘以$x$:
\[
xf'(x) \geq \frac{f(x+\eta x)-f(x)}{\eta}-\frac{C\eta}{2}
\]

综合上述两个不等式,我们得到:
\[
\frac{f(x+\eta x)-f(x)}{\eta}-\frac{C\eta}{2} \leq xf'(x) \leq \frac{f(x)-f(x-\eta x)}{\eta}+\frac{C\eta}{2(1-\eta)^2}
\]

现在考虑极限行为。由于$f \in C[0,+\infty)$,当$x \to 0^+$时:
\[
\lim_{x\to 0^+} \frac{f(x+\eta x)-f(x)}{\eta} = 0, \quad
\lim_{x\to 0^+} \frac{f(x)-f(x-\eta x)}{\eta} = 0
\]

设$L = \liminf_{x\to 0^+} xf'(x)$和$M = \limsup_{x\to 0^+} xf'(x)$。由上述不等式,取$x \to 0^+$,我们有:
\[
-\frac{C\eta}{2} \leq L \leq M \leq \frac{C\eta}{2(1-\eta)^2}
\]

由于这对任意$\eta \in (0,1)$都成立,令$\eta \to 0^+$,得到:
\[
0 \leq L \leq M \leq 0
\]

因此$L = M = 0$,即$\lim_{x\to 0^+} xf'(x) = 0$。
\end{proof}

为什么我们的条件看起来不是对称的呢?事实上,对于$f''(x)\geq- \frac{C}{x^2}$的情况,使用几乎完全相同的方法也能得到相同的结论。这表明,\textbf{无论是上界还是下界的二阶导数控制,都足以确保}$\lim_{x\to 0^+} xf'(x)=0$。\\
这个证明的深刻之处在于:\textbf{单侧控制足以扼杀震荡}。表面看,条件\\$f''(x)\leq\frac{C}{x^2}$只限制了$f$向上弯曲的程度,似乎允许$f$自由地向下剧烈震荡。但实际并非如此。函数要产生高频震荡(如振幅不衰减的振荡),其二阶导数必须在正负两个方向都具有足够大的振幅,以完成完整的波峰-波谷-波峰循环。单侧限制破坏了这种对称性,使震荡无法持续。

函数$f(x)=x\sin(1/x)$(定义$f(0)=0$)是一个绝佳的反例:它在$x\to0^+$时连续,但$xf'(x)=x\sin(1/x)-\cos(1/x)$不趋于0。计算可得其二阶导数的主项为$f''(x)\sim\cos(1/x)/x^3$,其振幅以$1/x^3$增长,正负交替,\textbf{突破了}无论是上界还是下界的二阶导数控制。这正说明$\frac{C}{x^2}$是临界尺度——弱于此尺度的控制无法抑制震荡,而达到此尺度的单侧控制已足以确保$xf'(x)\to0$。

\end{document}

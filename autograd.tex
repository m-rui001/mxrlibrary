\documentclass{beamer}
\usepackage[UTF8]{ctex}
\usepackage{amsmath}
\usepackage{amssymb}
\usepackage{graphicx}
\usepackage{booktabs}
\usepackage{tikz}
\usepackage{algorithm2e}
\usepackage{listings}
\usepackage{xcolor}
\usepackage{hyperref}

% Listings Settings
\lstset{
    basicstyle=\small\ttfamily,
    keywordstyle=\color{blue},
    commentstyle=\color{green!60!black},
    stringstyle=\color{red},
    showstringspaces=false,
    breaklines=true,
    frame=single,
    tabsize=4,
    numbers=left,
    numberstyle=\tiny\color{gray},
    extendedchars=true,
    inputencoding=utf8,
    literate={…}{{...}}3 {,}{{,}}1 {;}{{;}}1 {(}{{(}}1 {)}{{)}}1 {%}{{\%}}1 {!}{{!}}1 {:}{{:}}1,
}
\usetheme{Madrid}
\definecolor{myteal}{cmyk}{0.5,0,0.15,0}
\definecolor{myyellow}{cmyk}{0,0.2,0.7,0}
\definecolor{myblue}{cmyk}{0.80,0.13,0.14,0.04}
\definecolor{mygreen}{cmyk}{0.4,0,0.4,0}

\setbeamercolor{structure}{fg=mygreen}
\setbeamercolor{title}{fg=white, bg=mygreen}
\setbeamercolor{frametitle}{fg=white, bg=mygreen!90!black}
\setbeamercolor{block title}{fg=white, bg=mygreen!80!black}
\setbeamercolor{block body}{bg=mygreen!5}
\setbeamercolor{itemize item}{fg=mygreen}

\title{自动微分}
\institute{上海师范大学数理学院}
\begin{document}
\frame{\titlepage}
\begin{frame}{目录}
    \tableofcontents
\end{frame}
% ==================================================
\begin{frame}{什么是自动微分?}
\begin{block}{定义}
自动微分(AD)是一种基于链式法则和计算图的数值技术,能够精确高效地计算复杂函数的导数。
\end{block}

\begin{itemize}
\item 不同于数值微分(近似、易出错)和符号微分(表达式爆炸)
\item 现代深度学习框架(PyTorch, TensorFlow)的核心引擎
\item 实现了反向传播的底层机制
\item 支持高阶导数、控制流和自定义梯度
\item 使训练拥有百万级参数的神经网络成为可能
\end{itemize}

\begin{exampleblock}{核心价值}
当我们在PyTorch中调用\texttt{loss.backward()}时,系统能在数秒内计算出所有参数的梯度。自动微分技术是深度学习成功的关键基石。
\end{exampleblock}
\end{frame}

% ==================================================
\begin{frame}{微分方法比较}
\begin{block}{总结}
\begin{tabular}{|l|l|l|}
\hline
\textbf{方法} & \textbf{优点} & \textbf{缺点} \\
\hline
数值微分 & 简单易实现 & 精度低,效率差,误差大 \\
符号微分 & 精确结果 & 表达式爆炸,不支持控制流 \\
自动微分 & 精确+高效+完整代码支持 & 实现复杂 \\
\hline
\end{tabular}
\end{block}

\begin{itemize}
\item \textbf{数值微分}: $ f'(x) \approx \frac{f(x+h)-f(x)}{h} $
\begin{itemize}
\item 选择合适$h$困难:太小导致舍入误差,太大导致截断误差
\item $n$维参数计算成本为$O(n)$
\end{itemize}
\item \textbf{符号微分}: 代数推导 $ \frac{d}{dx}\sin(x^2) = 2x\cos(x^2) $
\begin{itemize}
\item 复杂表达式呈指数级增长
\item 无法处理条件语句或循环
\end{itemize}
\item \textbf{自动微分}: 将函数分解为基本运算
\begin{itemize}
\item 机器精度的准确性
\item 与原函数时间复杂度在同一量级
\item 支持任意Python控制流
\end{itemize}
\end{itemize}
\end{frame}

% ==================================================
\section{计算图与链式法则}
\begin{frame}{计算图}
\begin{block}{基本原理}
将函数表示为有向无环图(DAG),节点为变量/操作,边表示数据依赖关系。
\begin{exampleblock}{示例: $ y = \sin(x_1 x_2) $}
计算步骤:
\begin{enumerate}
\item $ v_1 = x_1 \times x_2 $
\item $ v_2 = \sin(v_1) $
\item $ y = v_2 $
\end{enumerate}

图结构:
\begin{itemize}
\item 输入节点: $x_1, x_2$
\item 中间节点: $v_1, v_2$
\item 输出节点: $y$
\item 边表示数据流向和依赖关系
\end{itemize}
\end{exampleblock}
\end{block}
\end{frame}

% ==================================================
\begin{frame}{链式法则}
\begin{block}{核心公式}
对于复合函数 $y = f(g(h(x)))$:
$$
\frac{dy}{dx} = \frac{dy}{dg} \cdot \frac{dg}{dh} \cdot \frac{dh}{dx}
$$
\end{block}

\begin{itemize}
\item \textbf{前向传播}: 计算函数值
    \begin{itemize}
    \item 按拓扑顺序执行操作
    \item 存储中间结果用于梯度计算
    \end{itemize}
\item \textbf{反向传播}: 从输出到输入传播梯度
    \begin{itemize}
    \item 利用已存储的中间结果
    \item 应用链式法则计算各层导数
    \end{itemize}
\end{itemize}
\end{frame}

% ==================================================
\section{自动微分的两种模式}
\begin{frame}{前向模式}
\begin{block}{工作原理}
\begin{itemize}
\item 从输入到输出传播导数
\item 在计算中间值的同时计算导数
\item 每次前向传播计算一个方向的导数
\end{itemize}
\end{block}

\begin{exampleblock}{数学表达}
对于 $f: \mathbb{R}^n \rightarrow \mathbb{R}^m$,前向模式计算:
$$
J \cdot \mathbf{v} = \begin{bmatrix}
\frac{\partial f_1}{\partial x_1} & \cdots & \frac{\partial f_1}{\partial x_n} \\
\vdots & \ddots & \vdots \\
\frac{\partial f_m}{\partial x_1} & \cdots & \frac{\partial f_m}{\partial x_n}
\end{bmatrix} \begin{bmatrix} v_1 \\ \vdots \\ v_n \end{bmatrix}
$$
\end{exampleblock}

\begin{itemize}
\item 适用于输入维度$n\ll$ 输出维度$m$的情况
\item 通常无需计算出Jacobian矩阵,每次前向传播计算一个方向的导数
\item 无需存储图
\end{itemize}
\end{frame}

% ==================================================
\begin{frame}{反向模式}
\begin{block}{工作原理}
\begin{itemize}
\item \textbf{前向阶段}: 计算值并记录计算图,\textbf{反向阶段}: 从输出到输入反向传播
\item 一次反向传播计算所有输入的梯度
\end{itemize}
\end{block}

\begin{exampleblock}{数学表达}
对于 $f: \mathbb{R}^n \rightarrow \mathbb{R}^m$,反向模式计算:
$$
\mathbf{v}^T \cdot J = \begin{bmatrix} v_1 & \cdots & v_m \end{bmatrix} \begin{bmatrix}
\frac{\partial f_1}{\partial x_1} & \cdots & \frac{\partial f_1}{\partial x_n} \\
\vdots & \ddots & \vdots \\
\frac{\partial f_m}{\partial x_1} & \cdots & \frac{\partial f_m}{\partial x_n}
\end{bmatrix}
$$
\end{exampleblock}
\begin{itemize}
\item 在调用 torch.autograd.grad 或 tf.gradients 时,若输出非标量,必须通过 grad\_outputs 显式指定 v,以表明你希望对哪些输出分量求导、以及各自的权重。
\item 空间复杂度: $O(\text{图大小})$,时间复杂度:$O(m)$
\end{itemize}
\end{frame}

% ==================================================
\begin{frame}{前向与反向模式}
\begin{block}{为什么反向模式主导深度学习?}
\begin{itemize}
\item 损失函数通常是标量输出($m=1$)
\item 模型拥有海量参数(百万到十亿级)
\item 前向模式计算所有梯度需要$O(n)$次计算
\item 反向模式仅需$O(1)$次计算(因为$m=1$)
\item 计算效率对比:
\begin{itemize}
\item 前向模式: $n$次前向传播
\item 反向模式: 1次前向+1次反向传播
\end{itemize}
\end{itemize}
\end{block}
\end{frame}

% ==================================================
\section{PyTorch autograd}
\begin{frame}[fragile]{PyTorch autograd基础}
\begin{block}{核心组件}
\begin{itemize}
\item \textbf{Tensor}: 基本数据结构,具有\texttt{requires\_grad}属性
\item \textbf{Function}: 记录操作历史,定义前向/反向计算
\item \textbf{Autograd引擎}: 构建计算图并执行反向传播
\end{itemize}
\end{block}

\begin{lstlisting}[language=Python]
import torch
x = torch.tensor(3.0, requires_grad=True)
w = torch.tensor(2.0, requires_grad=True)
b = torch.tensor(1.0, requires_grad=True)
y = w * x + b  # y = 7
loss = y**2    # loss = 49
loss.backward()
print(x.grad)  # dx = 28.0
print(w.grad)  # dw = 42.0
print(b.grad)  # db = 14.0
\end{lstlisting}
\end{frame}

% ==================================================
\begin{frame}{autograd内部机制}
\begin{block}{图构建}
\begin{itemize}
\item 运行张量运算时,都会生成一个对应的 Function 对象。这个对象负责记录这次运算的信息。
\item Function 对象不仅记录了参与运算的张量,还记录了进行的操作类型(如加法、乘法等)
\item 形成图,节点代表 Function 或者是张量,边表示计算顺序。
\item 叶节点(输入)存储梯度;中间节点通常不存储
\end{itemize}
\end{block}

\begin{block}{梯度计算规则}
\begin{itemize}
\item 每个\texttt{Function}定义\texttt{forward}和\texttt{backward}方法
\item \texttt{backward}接收上游梯度并计算下游梯度
\item 梯度累积在叶节点的\texttt{.grad}属性中
\end{itemize}
\end{block}
\end{frame}

% ==================================================
\begin{frame}[fragile]{autograd核心API}
\begin{block}{关键函数}
\begin{itemize}
\item \texttt{torch.autograd.grad(outputs, inputs)}:
\begin{itemize}
\item 显式计算梯度而不修改\texttt{.grad}
\item 支持高阶导数
\end{itemize}
\item \texttt{torch.autograd.functional.jacobian(function, inputs)}:
\begin{itemize}
\item 为多输出函数计算雅可比矩阵
\end{itemize}
\item \texttt{torch.autograd.functional.hessian(function, inputs)}:
\begin{itemize}
\item 计算海森矩阵(二阶导数)
\end{itemize}
\end{itemize}
\end{block}

\begin{lstlisting}[language=Python]
x = torch.tensor(2.0, requires_grad=True)
y = x**3
dy_dx = torch.autograd.grad(y, x, create_graph=True)[0]
d2y_dx2 = torch.autograd.grad(dy_dx, x)[0]
print(dy_dx)   # 12.0
print(d2y_dx2) # 12.0
\end{lstlisting}
\end{frame}

\begin{frame}[fragile]{多输出函数的混合梯度}
\begin{lstlisting}[language=Python]
import torch
x = torch.tensor(2.0, requires_grad=True)
y1 = x ** 2      # y1 = x^2
y2 = x ** 3      # y2 = x^3
# we want: d(y1)/dx  + d(y2)/dx 
# put outputs in a list,grad_outputs is the weights
grad_combined = torch.autograd.grad(
    outputs=[y1, y2],
    inputs=x,
    grad_outputs=[torch.tensor(1.0), torch.tensor(1.0)] 
)[0] #you can also use torch.ones_like([y1, y2])
print(f"Combined gradient = {grad_combined.item():.2f}")
# dy1/dx = 2x = 4
# dy2/dx = 3x^2 = 12
# weighted sum= 1*4 + 1*12 = 16
\end{lstlisting}
\end{frame}

% ==================================================
\begin{frame}[fragile]{梯度控制与优化}
\begin{block}{梯度管理}
\begin{itemize}
\item \texttt{with torch.no\_grad():}: 禁用梯度计算
\begin{itemize}
\item 推理和验证阶段必需
\item 减少内存使用,提高速度
\end{itemize}
\item \texttt{tensor.detach()}: 创建无梯度追踪的副本
\begin{itemize}
\item 阻止梯度流向特定张量
\end{itemize}
\item \texttt{requires\_grad\_()}: 动态启用/禁用梯度
\begin{itemize}
\item 支持微调: 冻结部分层,仅训练其他层
\end{itemize}
\end{itemize}
\end{block}

\begin{lstlisting}[language=Python]
with torch.no_grad():
    prediction = model(x_test) 
for param in model.parameters():
    param.requires_grad_(False)  
model.fc.requires_grad = True  #Two methods are similar,the above one is preferred
\end{lstlisting}
\end{frame}

% ==================================================
\begin{frame}[fragile]{梯度累积与清零}
\begin{block}{正确的梯度更新流程}
\begin{enumerate}
\item \textbf{清零梯度}: \texttt{optimizer.zero\_grad()}
\item \textbf{前向传播}: 计算预测和损失
\item \textbf{反向传播}: \texttt{loss.backward()}
\item \textbf{更新参数}: \texttt{optimizer.step()}
\end{enumerate}
\end{block}

\begin{block}{为什么需要清零梯度?}
\begin{itemize}
\item PyTorch \textbf{默认将梯度求和}(不覆盖)
\item 每次\texttt{backward()}调用将梯度添加到现有\texttt{.grad}值
\item 不清零会导致跨迭代的梯度错误累积
\end{itemize}
\end{block}
\end{frame}

\begin{frame}[fragile]{训练循环示例}
\begin{block}{标准训练步骤}
\begin{itemize}
\item 清零梯度
\item 前向传播
\item 计算损失
\item 反向传播
\item 更新参数
\begin{lstlisting}[language=Python]
for epoch in range(epochs):
    for batch in dataloader:
        optimizer.zero_grad() 
        outputs = model(inputs)
        loss = criterion(outputs, targets)
        loss.backward()
        optimizer.step()
\end{lstlisting}
\end{itemize}
\end{block}
\end{frame}

% ==================================================
\begin{frame}[fragile]{自定义梯度函数}
\begin{block}{torch.autograd.Function}
\begin{itemize}
\item 定义自定义操作及其梯度
\item 继承\texttt{torch.autograd.Function}
\item 实现静态\texttt{forward}和\texttt{backward}方法
\item \texttt{forward}: 定义计算
\item \texttt{backward}: 定义梯度计算
\end{itemize}
\end{block}
\end{frame}

\begin{frame}[fragile]{自定义自动微分函数}
\begin{lstlisting}[language=Python]
import torch
from torch.autograd import Function
class Square(Function):
    @staticmethod
    def forward(ctx, input):
        ctx.save_for_backward(input)
        return input ** 2
    @staticmethod
    def backward(ctx, grad_output):
        input, = ctx.saved_tensors
        grad_input = 2 * input * grad_output
        return grad_input
x = torch.tensor(3.0, requires_grad=True)
y = Square.apply(x)
y.backward()
print("x.grad =", x.grad)  
\end{lstlisting}
\end{frame}

% ==================================================
\begin{frame}{自动微分调试}
\begin{block}{常见问题与解决方案}
\begin{itemize}
\item \textbf{梯度消失/爆炸}:
\begin{itemize}
\item 监控梯度范数: \texttt{torch.norm(param.grad)}
\item 应用梯度裁剪: \texttt{torch.nn.utils.clip\_grad\_norm\_}
\end{itemize}
\item \textbf{缺失梯度(None)}:
\begin{itemize}
\item 验证\texttt{requires\_grad=True}已设置
\item 检查计算路径是否中断
\end{itemize}
\item \textbf{原地操作错误}:
\begin{itemize}
\item 避免修改参与梯度计算的张量
\item 优先使用\texttt{x = x + 1}而非\texttt{x += 1}
\end{itemize}
\end{itemize}
\end{block}

\begin{block}{调试工具}
\begin{itemize}
\item \texttt{torch.autograd.detect\_anomaly()}: 捕获异常操作
\item \texttt{torch.autograd.gradcheck()}: 数值验证梯度
\item 检查\texttt{tensor.grad\_fn}: 查看计算图节点
\end{itemize}
\end{block}
\end{frame}

% ==================================================
\begin{frame}[fragile]{ 检查点}
\begin{block}{内存优化}
\begin{itemize}
\item 问题: 大型模型存储计算图消耗过多内存
\item 前向时:只保存 layer2 的输入 x,不保存输出
\item 反向时:用保存的输入 x 重新运行 layer2 的前向
\item 权衡: 以额外计算换取内存减少
\end{itemize}
\end{block}
\begin{lstlisting}[language=Python]
from torch.utils.checkpoint import checkpoint
class MyModel(nn.Module):
    def __init__(self):
        super().__init__()
        self.layer1 = nn.Linear(100, 100)
        self.layer2 = nn.Linear(100, 100)
        self.layer3 = nn.Linear(100, 10)
    def forward(self, x):
        x = self.layer1(x)
        x = checkpoint(self.layer2, x)
        x = self.layer3(x)
        return x
\end{lstlisting}
\end{frame}

\end{document}